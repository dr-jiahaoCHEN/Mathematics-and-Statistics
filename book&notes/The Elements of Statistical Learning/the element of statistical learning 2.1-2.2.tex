\documentclass[UTF8]{ctexart}

\title{The Elements of Statistical Learning notes 2.1-2.2 part}

\author{Augest大魔王}

\date{\today}

\usepackage{hyperref}
\hypersetup{
	colorlinks=true,
	linkcolor=blue,
	filecolor=blue,      
	urlcolor=blue,
	citecolor=cyan,
}

\begin{document}  
	\maketitle
	
	本文是关于\href{https://github.com/dr-jiahaoCHEN/Mathematics-and-Statistics/blob/main/book%26notes/The%20Elements%20of%20Statistical%20Learning/(Springer%20Series%20in%20Statistics)%20Trevor%20Hastie%2C%20%20Robert%20Tibshirani%2C%20Jerome%20Friedman%20-%20The%20Elements%20of%20%20Statistical%20Learning_%20%20Data%20Mining%2C%20Inference%2C%20and%20Prediction.-Springer%20(2013).pdf}{《The Elements of Statistical Learning learning》}的学习笔记,主要涉及书的第一章到第二章的部分内容。该部分主要为入门介绍与基本概念的阐述。
	
	\section{介绍}
	\textbf{输入变量(inputs)},可以是测量得到或者预设的.这些变量对一个或多个\textbf{输出变量(outputs)} 有影响.便是利用输入变量去预测输出的值.这样的过程称之为**监督学习(supervised learning)**.
	
	在统计学中,\textbf{输入变量(inputs)}通常称\textbf{预测变量(predictors)},也可以被叫做\textbf{自变量(independent variables)}.在模式识别中,被叫做\textbf{特征(features)}的说法。
	
	\textbf{输出变量(outputs)}被称作\textbf{响应变量(responses)},也可以被叫做\textbf{因变量(dependent variables)}。
	
	
	\section{变量类型与术语}
	输出变量的类型,根据度量可以分为\textbf{定量的(quantitative)}与\textbf{定性的(qualitative)}。定性变量也被称为\textbf{类别型 (categories)}或者\textbf{离散(discrete)型变量},也被称作\textbf{因子(factors)}。
	
	对于两种类型的输出变量,考虑使用输入变量去预测输出变量是有意义的。当我们预测定量的输出时被称为\textbf{回归(regression)},当我们预测定性的输出时被称为\textbf{分类(classification)}.我们将会看到这两个任务有很多的共同点,特别地,两者都可以看成是**函数逼近**.
	
	输入变量也有各种类型,除了定性和定量以外,还有第三类\textbf{有序分类 (ordered categorical)},如 小(small)、中 (medium) 和 大 (large),在这些值之间存在顺序,但是没有合适的度量概念(中与小之间的差异不必和大与中间的差异相等).
	
	定性的变量常用数字编码来表示.最简单的情形是只有两个分类,比如说“成功”与“失败”,“生存”与“死亡”.这些经常用一位二进制数来表示,比如 0或 1,或者用-1 和 1来表示,这些编码有时被称作\textbf{指标 (targets)}.当存在超过两个的类别,存在其他可行的选择.最有用并且最普遍使用的编码是\textbf{虚拟变量(dummy variables)}.这里有 K个水平的定性变量被一个 K位的二进制变量表示,每次只有一个在开启状态,尽管更简洁的编码模式也是可能的,但虚拟变量在因子的层次中是对称的.
	
	我们将经常把输入变量用符号 $X$ 来表示. 如果 $X$ 是一个向量, 则它的组成部分可以用下标 $X_{j}$ 来取出. 
	
	定量的输出变量用 $Y$ 来表示, 对于定性的输出变量采用 $G$ 来表示. 
	
	我们] 使用大写字母 $X, Y, G$ 来表示变量, 对变量的观测值我们用小写字母来表示; 因此 $X$ 的第 $i$ 个观测值记作 $x_{i}$ 
	
	举个例子, $N$ 个 $p$ 维输入向量 $x_{i}, i=1, \cdots, N$ 可以表示成 $N \times p$ 的矩阵 $\mathbf{X}$. 
	
	\begin{equation}
		\mathbf{X}=\left[\begin{array}{cccc}
			x_{11} & x_{12} & \cdots & x_{1 p} \\
			x_{21} & x_{22} & \cdots & x_{2 p} \\
			\vdots & \vdots & \ddots & \vdots \\
			x_{N 1} & x_{N 2} & \cdots & x_{N p}
		\end{array}\right]
	\end{equation}
	
	通常我们把维度用列表示,观测值用行表示。因此,根据上述矩阵,每一行都是一个输入向量,每一列则是一个维度。
	
	我们简单定义统计学习,如下: 给定输入向量 $X$, 对输出 $Y$ 做出一个不错估计, 记为 $\hat{Y}$. 并且如果 $Y$ 取值为实数 $\mathbf{R}$, 则 $\hat{Y}$ 取值也是实数 $\mathbf{R}$; 同样地, 对于类别型输出, $\hat{G}$ 取值为对应 $G$ 取值的集合 $\mathcal{G}$.
	
\end{document}